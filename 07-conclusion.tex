\section{Conclusions and Future Work}

Scientific workflows are a \textit{de facto} standard for modeling scientific experiments. These workflows are usually composed of several black-box programs and data dependencies among programs. Such programs are executed many times during the execution of a workflow and produce many data files (\textit{i.e.,} the result of the workflow). When these workflows are executed in the cloud, the SWfMS commonly stores them in the same bucket. By doing this, the security of the research could be compromised. If an unauthorized user accesses those files, he/she may have some insights about the workflow results (and the workflow structure), which should be avoided.

In this article, we give a general description of both a simplified and a real-case problems, modeled as combinatorial optimization problems in order to preserve results confidentiality in cloud-based scientific workflows. We introduced two mathematical formulations and a heuristic procedure to solve these optimization problems. The proposed approach, named OPTIC, intend to generate a data distribution plan to disperse data files produced during the workflow execution, respecting data file conflicts and capacity constraints. 

Both mathematical models were solved by a high-performance mathematical programming solver, and the results indicated the effectiveness of the proposed approach, producing optimal distributions plans in a reasonable computational time, aiding cloud-based scientific workflows to preserve their results confidentiality. In cases where the solver did not produce a feasible distribution plan, the proposed heuristic has shown a simple, fast and effective alternative. Moreover, additional experiments revealed that the overhead derived from the integration of OPTIC with SWfMS is clearly acceptable, since it guarantees the confidentiality of the results.          

The proposed approach does not use fragmentation of data neither client-side encryption. Instead, OPTIC distributes the generated files across several buckets in cloud storages in order to avoid that an unauthorized user to access the files and infer some knowledge about the structure of the experiment and unpublished results.

\textcolor{red}{As a future work, we intend to extend the study on preserving results confidentiality by strengthening both mathematical models. Yet, implement and use other local and global search procedures in order to enhance solution quality of the proposed heuristic, specially those based on large neighborhoods \cite{Capua2018}. Another interesting future work might be the combination of the heuristic with data mining techniques, in which patterns extracted from good solutions can be used to guide the search in the solution space \cite{deLimaMartins2018}. Finally, we could consider financial costs involved in the process, \textit{i.e.,} OPTIC should minimize the  cost  of  a  workflow by  storing  data  across various  storage  platforms  depending  on  their  size.}







We introduce our science gateway \system as a user-friendly interface that manages executions of bioinformatics applications with HPC/HTC technologies in computational resources, including clusters of supercomputers. \system is implemented over the grid middleware CSGrid and uses the computational resources of the Brazilian network SINAPAD, which bring-out grid computing management services. The experiences of SINAPAD at developing science gateways demonstrated the feasibility of the use of the technologies HPC/HTC [9].
Data analytics in science gateways are essential to support the exploratory nature of science. Large-scale experiments can benefit from data analytics facilities to evaluate the results, reduce the incidence of errors, decrease the total execution time, and sometimes reduce the financial cost. The data analysis process in science gateways needs to explore the statistics of applications execution, performance issues, and the content of data files. Each application execution in science gateways may consist of hours or days of processing, thus, it is unfeasible to perform an analysis without automatic semantic computational database support.
This paper evaluates the architecture of BionfoPortal, coupled with technologies of management systems and HPC resources to explore the processes of bioinformatics applications in an efficient manner. We are also concerned of coupling to \system with the best configurations for the efficient use of the computational resources, especially for the MPI and multithreading applications RAxML, SPAdes, FragGeneScan, MAFFT, Ray, Bowtie, and HMMER. However \system is free available and functional, there are several open challenges concerning to the maintenance of security in science gateways and coupling management systems for databases and workflows, which can be supported by machine learning technologies.
