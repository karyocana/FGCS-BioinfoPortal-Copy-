\section{Related Work}

This section aims at presenting a review of the literature in the topic of science gateways. Many science gateways offer sharing possibilities within a community using technologies based on the reusability of methods and reproducibility of diverse fields of science [7]. There are several initiatives of bioinformatics research groups for developing web interfaces, portals or migrating workflows or software for the community. We highlighted for discussion some main science gateways that covered HPC technologies.

MoSGrid [10] is a web-based science gateway for structural bioinformatics that provides an intuitive user interface to several applications. The security concept applies SAML (Security Assertion Markup Language) and allows for trust delegation from the user interface layer, middleware layer, and Grid middleware layer with HPC facilities. 

Galaxy [11] is an intuitive science gateway that supports a large number of communities with overlapping research fields. However, Galaxy lacks the support of grid-based Distributed Computing Infrastructures (DCIs) and requires the installation of a Galaxy instance per underlying DCI. Thus, the migration of Galaxy workflows to WS-PGRADE workflows allows for flexibly using existing Galaxy workflows for various DCIs.

myExperiment \cite{10.1093/nar/gkq429} is an online research environment that supports the social sharing of bioinformatics workflows. As a public repository of workflows, myExperiment allows every user to discover those that are relevant to their research, which can then be reused and repurposed to their specific requirements. Although myExperiment represents a step forward, it does not allow users to execute their workflows.

The CIPRES\footnote{\url{https://www.phylo.org}} Science Gateway is a public resource for inference of large phylogenetic trees. It is designed to provide access to NSF XSEDE’s large computational resources through a simple browser interface. It released a RESTful API to allow integration of CIPRES capabilities into other desktop software and web applications. The CIPRES Science Gateway is designed to manage data much like an e-mail client. The data is then used to stage individual jobs.

