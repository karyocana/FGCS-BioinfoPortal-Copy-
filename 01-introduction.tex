\clearpage
\section{Introduction}
Nowadays, genomics research shows an unprecedented effort in sequencing and categorizing genomes produced by new-generation high-throughput DNA sequencing \cite{HEATHER20161}. The capacity for the biological data generation has led to an explosive growth of the complexity, heterogeneity, volume, and geographic dispersion of this biological “big data” \cite{10.1093/bib/bbx044}. Considering the annual growth of the generated data, it is estimated that the biological big data will reach 44 zettabytes in 2020 \cite{10.1093/bioinformatics/btx468}. Thus, analyzing this volume of data is far from trivial. The integration of the latest breakthroughs in biomedical technology from one side and High Performance Computing (HPC), High Throughput Computing (HTC), Scientific Workflow Management Systems (SWfMS) \cite{2002:WMM:509042}, and Database Management Systems (DBMS) \cite{Ramakrishnan:2003:DMS} from another side, enables remarkable advances in the fields of healthcare, drug discovery, genome research, computational biology, system biology, data science (management, sharing, and execution) and so on. 

Computational biology and bioinformatics are interdisciplinary fields that deal with the development of computational, mathematical, and biostatistics methods to analyze large biological datasets to infer hypotheses or discover new solutions. They have emerged as a very promising area in the analysis of genome sequences. Latin America has a very active research community interested in developing and using bioinformatics approaches for supporting academic, scientific, and industrial demands. The Brazilian Bioinformatics Network (RNBio) aims at strengthening the bioinformatics research projects in Brazil in a multi-institutional format with the training of specialized human resources in thematic studies involving bioinformatics and computational biology. RNBio has scientific collaborations with the Brazilian National System for High Performance Computing\footnote{\url{https://www.lncc.br/sinapad/}} (SINAPAD), which offers to users several heterogeneous and geographically distributed resources with high performance/throughput computing (HPC/HTC) capabilities and customized security models.

A science gateway \cite{GESING2018544} is defined as “a community-developed set of tools, applications, and data that is integrated via a portal or a suite of applications” \cite{Gesing:2016:USG:3037851.3037858}. SINAPAD applies grid computing \cite{doi:10.1177/109434200101500302} using the middleware CSGrid\footnote{\url{https://jira.tecgraf.puc-rio.br/confluence/display/CN/CSGrid+Home}} that offers the single access points through web interfaces (gateways) for the submission of scientific applications for the use of HPC resources. CSGrid offers two key entry points for users, a desktop Java client and a service bus (called OpenBus). OpenBus allows managing services as jobs (OpenDreams) and files (HDataService), provided as APIs. Other CSGrid services that can be accessed by science gateways are the mc2toolset, a RESTful web service (APIs to develop gateways), and a CLI (Command Line Interface). 

CSGrid has been used in different projects, as the mc2toolset that supports the prototyping of several scientific portals at SINAPAD for physics, meteorology, chemistry, complex networks, mathematical, bioinformatics, medicine, etc \cite{doi:10.1002/cpe.3258}. However, although CSGrid and their tools represent a step forward, it may be complex to the regular users to execute their experiments, since it may require advanced knowledge of several tools, frameworks, applications, filesystems, etc. \system invokes the middleware CSGrid for managing the requests of users for executing applications in the HPC/HTC resources of SINAPAD. At the end of the execution, a link pointing to the scientific results is sent to the e-mail of the users. 

The main goal of this paper is to address the problems and features of designing a multiuser computational platform for parallel/distributed executions of large-scale bioinformatics applications using the Brazilian HPC resources. A grid-based architecture for the science gateway is introduced. This architecture allows scientists to design and integrate components of a heterogeneous architecture for their experiments. Other goals of the Project called \system are: 1) to integrate HPC, SWfMS, and DBMS technologies; 2) to dispatch, manage, and execute processes in a transparent and friendly manner for (non)expert users; 3) to evaluate the performance of applications; and 4) to manage the provenance data tracking of records of executions at \system.

The results (scientific/performance) of HPC application executions at \system were analyzed with the support of the provenance databases, by submitting high level database analytical queries. These results show that \system is functional and capable of processing up to the datasets required for each one of the bioinformatics applications, especially HPC applications as RAxML \cite{raxmlVIII2014} with multithreads and MPI, which improved the performance. 

First, we analyzed the general features of the application executions (input size, software parameters, efficiency of machines capacity) in order to provide information about the best HPC resources utilizations to users. Then, we further explore the features of the allocation/usage of computational resources for the gateway using machine learning techniques. Decision trees generated with regression models, based on a historic of the dataset, provided a promisor learning module and proved that choosing the platform configuration for performing executions is valuable for exploring the better usage of the HPC infrastructure in science gateways.

The science gateway \system mediates the execution of a suite of bioinformatics applications using the computational resources of SINAPAD. It follows the Software as a Service (SaaS) delivery model and is built on top of the middleware CSGrid, which allows managing the bioinformatics applications (programs, tools, or workflows) with HPC/HTC. \system is currently supported by the team of RNBio, National Laboratory of Scientific Computing\footnote{\url{https://www.lncc.br/}} (LNCC) and SINAPAD and it is freely available for the academic use at \url{https://bioinfo.lncc.br/}.

This paper is organized as follows. Section II presents related works. Section III describes the architecture of the science gateway \system and its implementation using CSGrid and HPC resources of SINAPAD. Section IV shows the experimental results and discussion and Section V concludes the paper and points out future work.



